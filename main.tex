\documentclass{article}
\usepackage[utf8]{inputenc}

\title{Mensa HiQ: De oorsprong van alles.}
\author{Josko de Boer (NL7725)}
\date{December 2016}
\usepackage[square,numbers]{natbib}
\bibliographystyle{abbrvnat}
\usepackage{fourier}
\usepackage{multicol}
\usepackage{fullpage}
\usepackage[dutch]{babel} 
\usepackage[square,numbers]{natbib}
\bibliographystyle{abbrvnat} 
\usepackage{mathrsfs}
\usepackage{amsthm,amsmath,amssymb,amsfonts}  
\usepackage{braket} 
\usepackage{subcaption}
\usepackage{xcolor}
\usepackage{wrapfig}
\usepackage{graphicx}
\renewcommand{\thefootnote}{\Roman{footnote}}
\newcommand{\red}[1]{ {\color{red} #1}}
\date{February 2017} 
\begin{document}
\maketitle
\begin{abstract} 
Een korte introductie in de wetenschappelijke theorie\"en omtrent de oorsprong van alles. Josko de boer is opgeleid als theoretisch kwantum natuurkundige. Momenteel is hij niet-onderzoeker in het laboratorium van Prof. dr. Nynke Dekker, waar hij zich bemoeit met de data analyse van de onderzoekers.
\end{abstract} 
    \section{Introductie}        
        De wetenschap is een zoektocht naar kennis over hoe de natuur werkt. De wetenschap onderscheidt zich van de Filosofie door het rigoureuze experiment. Experimenten zijn primair in deze zoektocht. Vaak wordt er gebruik gemaakt van wiskunde, omdat wiskunde als deductief gereedschap ervoor zorgt dat we onze uitleg kunnen omvormen tot een voorspelling, om deze met het rigoureuze experiment te bevestigen.   
        
        Sir Isaac Newton formuleerde de vier regels van het wetenschappelijk redeneren in zijn \emph{Principia Mathematica}. De vierde luidt dat een bewering welke uit observatie stamt als accuraat moet worden beschouwd totdat het tegendeel geobserveerd is. Een wetenschappelijke theorie is een groep beweringen welke uitleg geeft aan een verschijnsel. Daarnaast zijn de voorspellingen van deze theorie bevestigd in een experiment en zijn deze te onderscheiden van andere theorie\"en, veelal door een `cruciaal' experiment. Het woord 'wet' wordt meestal gebruikt om een trend in de data aan te duiden; zo is de ideale gaswet in de meeste omstandigheden een accurate relatie tussen druk, volume en temperatuur.
        
        Ik zal chronologisch en beknopt de wetenschappelijke theori\"en van `de oorsprong van alles' uiteenzetten. Van de oerknal \cite{oerknal} tot de formatie van sterren \cite{ster} en planeten \cite{planeet}, de oorsprong van leven \cite{leven} en de gemeenschappelijke afstamming van alle huidige soorten \cite{afstamming}.
    \section{De oorsprong van het Universum}
        Een van de meest bekende wetenschappers was Albert Einstein. Een van zijn grote bijdrages is de \emph{Algemene Relativiteitstheorie} (AR), welke zwaartekracht verklaard als de kromming van ruimte-tijd en zoals recent gevierd en verder bewezen door observatie van zwaartekracht golven \cite{ligo}. 
        
        Het universum kan grootschalig beschreven worden met deze theorie\"en. Het \emph{kosmologisch principe} is de aanname dat materie gelijk verdeeld is over het universum. Combinatie met AR leidt tot de \emph{Friedmann vergelijkingen}, welke drie mogelijke vormen van het universum beschrijven.
        
        Met de verschuiving-wet van Wien kunnen we de golflengte (kleur) waarop het meeste licht wordt uitgezonden door een ster aan zijn temperatuur relateren. Echter gaven andere metingen voor de temperatuur van een ster een ander antwoord. Het verschil heet roodverschuiving en wordt verklaard omdat het universum \emph{uitdijt}. De snelheid waarmee dit gebeurt heet de Hubbleconstante. Een universum dat uitdijt is een van drie vormen van Friedmann.
        
        Toen wetenschappers terug in de tijd redeneerden kwam het inzicht dat het kleinere volume van het vroege universum zal leiden tot een veel hogere temperatuur. Zoals een explosie zal de uitzetting in eerste instantie veel sneller zijn. Hiervan stamt de naam van het model: \emph{De oerknal}. Dit leidt ook tot de bevestigde voorspelling van kosmische achtergrondstraling. Andere voorbeelden van bewijs voor dit model zijn de ratio 's van helium, deuterium en lithium, de vormen en de verdeling van galactische stelsels en het bestaan van primordiale gaswolken.
        
    \section{De oorsprong van sterren en planeten}         
        Primordiale gaswolken bestaan met name uit waterstof atomen. Deze trekken elkaar aan door zwaartekracht. De atomen botsen, wat hitte genereerde. Voor zo 'n gaswolk zijn er zoveel atomen dat deze toch blijft krimpen. Uiteindelijk is de zwaartekracht zo hoog dat de elektronen van de atomen afgebroken worden, wat wederom hitte produceert. Zelfs nu zal zwaartekracht de protonen bij elkaar trekken in een proces van nucleaire fusie, waarbij twee waterstofatomen een enkel helium atoom vormen en een grote hoeveelheid energie. Deze energie lekt langzaam weer weg; het beschreven object is namelijk een zon.
        
        Over zo 'n vijf miljard jaar zal onze zon radicaal veranderen. Wanneer de waterstof kern compleet gefuseerd is tot helium, zal de kern geen hitte meer produceren en daardoor krimpen. Hierdoor worden lagen verder naar buiten dichter, en zal hier ook fusie optreden. De hitte die hierbij vrijkomt laat de zon opzwellen tot een Rode Reus, een gigantische rode zon welke waarschijnlijk groot genoeg is dat de aarde opgeslokt wordt.
        
        De levenscyclus van sterren bevat vele fasen waarin zwaartekracht het uiteindelijk wint van de hitte-opwekkende processen. Hierdoor verschijnen de scheikundige elementen. Voor primordiale gaswolk die draaien zullen zwaardere elementen naar de buitenkant gecentrifugeerd worden. Het geheel ziet eruit als een schijf.   
        
        Deze protoplanetaire schijf bevat de ingredi\"enten voor een sterrenstelsel zoals het onze. De planeten vormen doordat de materie in de schijf elkaar ook aantrekt en langzaam klompen vormt. De klompen botsen waarbij soms twee klompen een worden, een proces dat uiteindelijk de planeten vormt. Energie-behoud dicteert dat de planeten beginnen te draaien zoals de eerdere schijf. Voor een planeet kan dit een magnetisch veld opleveren, wat vervolgens geladen deeltjes uit de schijf naar binnen trekt. Met daarbij het het wegzakken van de zwaardere delen van de protoplaneet zal de kern opwarmen. De planeet kan tektonische platen en vulkanisme ontwikkelen.
    
    \section{De oorsprong van leven}
        De oorsprong van level is innig verbonden met de scheikunde van koolwaterstoffen. Koolwaterstoffen kunnen lange ketens vormen van kool- en waterstof atomen. Dit gebeurt ook zonder het toevoegen van enzymen, zoals de Miller-Urey experimenten lieten zien. Inmiddels is hier meer onderzoek naar, bijvoorbeeld door het lab van Dr. Jack Szostak, Nobelprijs winnaar genetica. 
        
        De meest gangbare hypothese wordt de RNA wereld genoemd. Centraal in de RNA wereld staan twee simpele molecuul-soorten, welke in een wereld zonder leven geproduceerd kunnen worden. De eerste zijn amphifiele moleculen, welke eruit zien als een soort kikkervisje; de staart houdt van water, maar de kop niet. Opgelost in water steken deze stoffen de koppen bij elkaar. Afhankelijk van de zuurtegraad (pH) vormen ze bolletjes (micelles) of holle bollen (vesicles).
        
        De andere molecuuls-soort is een simpelere vorm van DNA genaamd RNA. De vorm van een (lang) molecuul heeft een elektrisch veld; rondzwevende kleinere moleculen reageren hierop. RNA is redelijk speciaal omdat deze vaak kleine moleculen 'goed' draait, zodat ze makkelijker met elkaar reageren. Dit heet katalyse, het versnellen van een scheikundige reactie.
        
        Evenals DNA bestaat RNA uit een suiker die als schakel werkt en daarnaast een soort markering per schakel. DNA zit meestal in de beroemde dubbele helix. RNA zit dit normaliter niet, maar kan dit wel. Het tweede molecuul van de dubbele helix kan zich spontaan schakel voor schakel vormen. Ook dit proces kan RNA versnellen, wat zelf-katalyse genoemd wordt. 
        
        In de RNA wereld hypothese zijn er vesicles met daarin RNA: een soort protocel. Nadat RNA spontaan een dubbele helix vormt (replicatie) splitst de RNA helix zich weer door bijvoorbeeld verhitting, wat ook de vesicle ook in twee\"en deelt. Hierdoor deelt het RNA en de vesicle, een soort protocel-deling.
        
        Belangrijk is dat RNA moleculen afhankelijk van hun letter-reeks een vorm aannemen. De katalytische activiteiten zijn dan ook afhankelijk van de letter-reeks. Deze reeks wordt gekopie\"erd door het zelf-katalyse proces, maar met fouten. Dit geeft erfbaarheid van activiteit en een zekere natuurlijke variatie. Dit zijn de twee ingredi\"enten van het verhaal van Charles Darwin, welke ook van toepassing is op de RNA wereld, zodat protocellen in cellen evolueren. 
    
        Charles Darwin publiceerde in 1859 het boek \emph{The Origin of Species by means of Natural Selection}. Het combineerde de twee ingredi\"enten met het strijden om gelimiteerde grondstoffen, zeker binnen een `soort'. Omdat sommige individuen simpelweg beter aangepast zijn, zullen deze zich vaker voortplanten. Hun nakomelingen zullen meestal deze eigenschappen erven. Hierdoor zal de soort langzaam beter worden in zo 'n aspect - dit loopt totdat een ander aspect een groter effect zal hebben. Zo zullen antilopes niet harder gaan rennen; er is een fragiele balans tussen de huidige snelheid en het oplopen van verwondingen. 
        
        De voorspellingen van deze theorie zijn gevarieerd en geobserveerd, niet alleen door fossielen maar ook door experimenten in het laboratorium. Het verklaard de differentiatie van verschillende soorten. Uiteindelijk zullen groepen niet meer met elkaar in contact komen, wat meestal resulteert in dat groepen niet meer gezamenlijk voort kunnen planten. Dit soort groepen noemen wij dan ook meestal `soorten'.
        
        We kunnen dit terug traceren tot zeker de eerste soorten met een skelet. Voor eerdere soorten is er simpelweg weinig bewijs dat de honderden miljoenen jaren heeft overleefd. Door de verschillende vormen van bewijs hebben wij een redelijk beeld van de evolutie, en de verwachting is dat dit alleen maar toe zal nemen.
    \section{Conclusie}
        Vanaf de oerknal, de formatie van sterren, planeten en uiteindelijk leven tot de weelde aan biodiversiteit die we nu kennen. Ieder onderdeel is kort beschreven en een minuscule hoeveelheid van het bewijs is aangestipt. 
        
        Vele onderzoekers hebben zich aan dit onderwerp gewaagd. Bijvoorbeeld Richard Dawkins met \emph{The Greatest Show on Earth} of Stephen Hawking met \emph{A brief History of Time}. De meeste werken zijn uitgebreider dan dit artikel met een sterkere nadruk op de expertise van de onderzoeker.
        
    \bibliography{referenties}  
\end{document}
